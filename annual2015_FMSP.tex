%%--------------- Text starts from here ----------- %%

%%%%%%%%%%%%%%%2013年度FMSP Annual Report用スタイルファイル%%%%%%%%%%%%%%%%%%%%%%%%

%%%%%%%%%%%%%%%%%%%%%%%%%%%%%%%%%%%%%%%
%留学生などはこのレポートを英文で作成して結構です。
%This report can be written in English as well.
%%%%%%%%%%%%%%%%%%%%%%%%%%%%%%%%%%%%%%%

%%%%%%%%%%%%%%%%%%%%%%%%%%%%%%%%%%%%%%%
%このFormatはpLaTex を使用しています。
%
%以下に報告書の基本形が示してありますので、参考にしてお書き下さい。 
%数字は2桁以上は全て半角で書いて下さい。 
%文末の空白は必ず半角でお願いします。全角の空白は TeX では特殊文字と 
%判断して問題を起こすことがあります。
%数式はかならずmath mode でお願いいたします。 
%事務局では校正をせずにコンパイルしたファイルをそのまま印刷に出しますので、 
%一度コンパイルして、スペルチェック、校正は必ず行なって下さい。 
%まとめの編集の都合上、\newcommand, \renewcommand, \def の追加等はさけて下さいま
%すようお願いいたします。
%参考文献を入れる場合は \cite などは用いず、文献番号を手で入れて下さい。
%%%%%%%%%%%%%%%%%%%%%%%%%%%%%%%%%%%%%%%

%%%%%%%%%%%%%%%%%%%%%%%%%%%%%%%%%%%%%%%%%
%このファイルは \input によってまとめてコンパイルします。
%ファイル名は次の例のように
% 学年_lastname_firstname.tex
% として下さい。
%(例) m1_suri_taro.tex
%該当する事項がない部分は項目ごと削除して下さい。
%%%%%%%%%%%%%%%%%%%%%%%%%%%%%%%%%%%%%%%%%%


\documentclass[a4j,twocolumn]{jarticle}

\usepackage{amssymb,amsmath}
\textheight=25cm
\textwidth=15cm
\parskip=0mm
\parindent=0mm
\topmargin=-1cm
\oddsidemargin=5mm

\begin{document}

%%%%%%%%%%%%%%%%%%%%%%%%%%%%%%%%%%%%%%%%%%
%%%----------- CUT HERE -----------------------------------------------------------------
%%%%%%%%%%%%%%%%%%%%%%%%%%%%%%%%%%%%%%%%%%


%%%%%%%%%%%%%%%%%%%%%%%%%%%%%%%%%%%%%%%%%%%%%%%%%%%%%%%%%%%%%%%%%%%%%%%%%%%% 
% 氏名(ローマ字綴りで名字は全て大文字,名前は最初の字だけ大文字) 
% を書いて下さい. {\bf 数理 太郎 (SURI Taro)}
%%%%%%%%%%% 
%%%%%%%%%%%%%%%%%%%%%%%%%%%%%%%%%%%%%%%

{\bf 高橋 和音 (TAKAHASHI Kazune)}


%学振DC1,国費などの留学生などに採用されている人は記載して下さい.
%(例) 学振DC1
%%%%%%%%%%%%%%%%%%%%%%%%%%%%%%%%%%%%%%%%%%%%%%%%%%%%%%%%%%%%%%%%

%%%%%%%%%%%%%%%%%%%%%%%%%%%%%%%%%%%%%%%%%%%%%%%%%%%%%%%%%%%%%%%%%%%%%%%%%%% 
% 所属専攻名と学年を記入して下さい
%%%%%%%%%%%%%%%%%%%%%%%%%%%%%%%%%%%%%%%%%%%%%%%
%(例) 数理科学専攻 修士課程1年
%物理学専攻 博士課程1年
%%%%%%%%%%%%%%%%%%%%%%%%%%% 

数理科学専攻 博士課程1年


\vspace{0.2cm}
\noindent
{\bf 研究概要}

\vspace{0.1cm}
%%%%%%%%%%%%%%%%%%%%%%%%%%%%%%%%%%%%%%%%%%%%%%%%%%%%%%%%%%%%%%%%%%%%%%%%%% 
%研究の要約を記入してして下さい.
%留学生の人などは英文でも結構です.
%コンパイルして0.5ページ以上2ページ以内程度になるようにまとめて下さい。
%%%%%%%%%%%%%%%%%%%%%%%%%%%%%%%%%%%%%%%%%%%%%%%%%%%%%%%%%%%%%%%%%%%%%%%%%%

I study the existence and nonexistence of the solutions
of elliptic PDEs using the variational method.
Last academic year I worked on the following
nonhomogeneous semilinear elliptic equation
involving the critical Sobolev exponent:
$-\Delta u + a u = b u^p + \lambda f$. Especially, I studied
the relationship between the dimension of the domain and
the existence and nonexistence of the solutions.
I proved that provided 
$b$ achieves its maximum at an inner point of the
domain and $a$ has a growth of the exponent $q$
in some neighborhood of that point, then
if the dimension of the domain is less than $6 + 2q$,
there exist at least two positive solutions.
It seems to be new that the coefficient of a linear term affects
the dimension of the domain on which solutions exist. 

\vspace{0.2cm}
\noindent
{\bf 発表論文}

\vspace{0.1cm}
%%%%%%%%%%%%%%%%%%%%%%%%%%%%%%%%%%%%%%%%%%%%%%%%%%%%%%%%%%%%%%%%%%%%%%%%%%%%%% 
% プレプリントも含めて,大学院進学後に発表したものをすべて書いて下さい。
%プレプリントarchiveに投稿したものは番号を記載して下さい。
% 様式は以下の例のように
% 著者・共著者名・ 題名・ジャーナル名・巻・年・ページの順に書いて下さい.
% タイトルの前に著者・共著者名を入れる形です。
% 共著の場合は著者名をすべて書いて下さい。
%%%%%%%%%%%%%%%%%%%%%%%%%%%%%%%%%%%%%%%%%%%%%%%%%%%%%%%%%%%%%%%%%%%%%%%%%%%%% 

\begin{enumerate}
\item K. Takahashi: \lq\lq Semilinear elliptic equations with critical
      Sobolev exponent and non-homogeneous term",
      Master Thesis, The University of Tokyo (2015).
\end{enumerate}

\vspace{0.2cm}
\noindent
{\bf 口頭発表}

\vspace{0.1cm}
%%%%%%%%%%%%%%%%%%%%%%%%%%%%%%%%%%%%%%%%%%%%%%%%%%%%%%%%%%%%%%%%%%%%%%%%%%%%%%%
% 大学院進学後に行なった研究発表について
% 昨年度以前のものも含めて
% タイトル・シンポジウム(またはセミナー等)名・場所・月・年を 
% 書いて下さい.国際会議の場合は国名をお願いします.タイトルは原題で.
%%%%%%%%%%%%%%%%%%%%%%%%%%%%%%%%%%%%%%%%%%%%%%%%%%%%%%%%%%%%%%%%%%%%%%%%%%%%%%% 

\begin{enumerate}
\item Semilinear elliptic equations with critical Sobolev exponent and non-homogeneous term, RIMS Workshop: Shapes and other properties of solutions of PDEs, RIMS, Kyoto University, Japan, Nov 2015.
\end{enumerate}

%\vspace{0.2cm}
%\noindent
%{\bf FMSPの活動への参加}

%\vspace{0.1cm}
%%%%%%%%%%%%%%%%%%%%%%%%%%%%%%%%%%%%%%%%%%%%%%%%%%%%%%%%%%%%%%%%%%%%%%%%%%%%%% 
% 本年度のFMSPが主催,共催する研究会,ワークショップ,FMSP Lectures
% などへの参加を予定も含めて記入して下さい.また,それ以外に本年度
% FMSPから旅費等の補助を得て参加した国内外の研究会,セミナーなどの参加も記載して下さい.
%また,このような活動への参加によって,どのようなことが得られたかを簡潔に書いて下さい.
% FMSPプログラムに参加の後に長期海外渡航、インターンシップなどを行なった
% 人は昨年度以前のものも含めて記入し下さい。また、長期海外渡航、インターンシップ
% でどのような活動を行なったか、どのような成果が得られたかについて数行で記述して
% 下さい。
%%%%%%%%%%%%%%%%%%%%%%%%%%%%%%%%%%%%%%%%%%%%%%%%%%%%%%%%%%%%%%%%%%%%%%%%%%%%%%%



\vspace{0.2cm}
\noindent
{\bf 受賞}

\vspace{0.1cm}
%%%%%%%%%%%%%%%%%%%%%%%%%%%%%%%%%%%%%%%%%%%%%%%%%%%%%%%%%%%%%%%%%%%%%%%%%% 
% 修士課程進学以降にありましたら書いて下さい. 研究科長賞などを含みます.
% 受賞年度を記入して下さい。
%%%%%%%%%%%%%%%%%%%%%%%%%%%%%%%%%%%%%%%%%%%%%%%%%%%%%%%%%%%%%%%%%%%%%%%%

\begin{enumerate}
 \item Code Runner 2015, Final Round: 1st place, Recruit Career, Tokyo, Dec 2015.
 \item SamurAI Coding 2014-15, World Final: 6th place, 77th Information Processing Society of Japan National Convention, Kyoto University, Japan, Mar 2015.
 \item Code Runner 2014, Final Round: 7th place, Recruit Career, Tokyo, Nov 2014.
 \item Code Festival 2014 AI Challenge, Final Round: 3rd place, Recruit Holdings, Tokyo, Nov 2014.
\end{enumerate}

\vspace{0.4cm}

%%%%%%%%%%%%%%%%%%%%%%%%%%%%%%%%%%%%%%%%%%
%%%----------- CUT HERE -----------------------------------------------------------------
%%%%%%%%%%%%%%%%%%%%%%%%%%%%%%%%%%%%%%%%%%

\end{document} 
